\documentclass[11pt]{article}
\usepackage{calc,fancyhdr,lastpage}
\usepackage{graphicx}
\usepackage[hmargin=.75in,vmargin=1in,
            footskip=.55in,headsep=.55in-\headheight]{geometry}
\usepackage{amsmath,amssymb}

% Formats, symbols, abbreviations.
\let\altemph\textsl
\let\strong\textbf
\let\code\texttt
\let\latinabb\emph
\newcommand*{\etc}{\latinabb{etc}}
\newcommand*{\eg}{\latinabb{e.g.}}
\newcommand*{\ie}{\latinabb{i.e.}}
% To get proper-looking symbols in \texttt.
\newcommand*{\txtbksl} {\symbol{"5C}}% \
\newcommand*{\txtcaret}{\symbol{"5E}}% ^
\newcommand*{\txtunder}{\symbol{"5F}}% _
\newcommand*{\txtlcurl}{\symbol{"7B}}% {
\newcommand*{\txtrcurl}{\symbol{"7D}}% }
\newcommand*{\txttilde}{\symbol{"7E}}% ~

%% A heading in the instructions.
\newcommand*{\heading}[1]{\subsubsection*{#1}}

% Headings.
\pagestyle{fancy}
\let\headrule\empty
\let\footrule\empty
\lhead{{\bfseries CSC\,320\,H1S}}
\chead{{\bfseries\large Assignment \#\,4
    --- General Instructions}}
\rhead{{\bfseries Winter 2015}}
\lfoot{{Dept. of Computer Science, University of Toronto}}
\cfoot{{}}
\rfoot{{Page \thepage\ of \pageref{LastPage}}}


\title{CSC320 A4}
\author{Maxwell Huang-Hobbs (g4rbage)\\Alexander Biggs(g4biggse)}
\date{March 2015}

\begin{document}

\maketitle
\begin{center}
Starting with:  
$$C=F + (1-\alpha)B$$
expand and rearrange:\\
$$C=F + B- \alpha B$$
$$C-B=F - \alpha B$$
Because the above relationship holds for all colour channels of both images, this gives us the system
$$\begin{pmatrix}
C^{r}_1 - B^{r}_1 = F^{r} -\alpha B^{r}_1\\
C^{g}_1 - B^{g}_1 = F^{g} -\alpha B^{g}_1\\
...\\
C^{g}_2 - B^{g}_2 = F^{g} -\alpha B^{g}_2\\
C^{b}_2 - B^{b}_2 = F^{b} -\alpha B^{b}_2\\
\end{pmatrix}$$
or
$$\begin{pmatrix}
  C^{r}_1\\
  C^{g}_1\\
  C^{b}_1\\
  C^{r}_2\\
  C^{g}_2\\
  C^{b}_2\\
\end{pmatrix}
-\begin{pmatrix}
  B^{r}_1 \\
  B^{g}_1 \\
  B^{b}_1 \\
  B^{r}_2 \\
  B^{g}_2 \\
  B^{b}_2 \\
\end{pmatrix}
=
\begin{pmatrix}
  F^{r} \\
  F^{g} \\
  F^{b} \\
  F^{r} \\
  F^{g} \\
  F^{b} \\
\end{pmatrix}
-
\alpha
\begin{pmatrix}
  B^{r}_1 \\
  B^{g}_1 \\
  B^{b}_1 \\
  B^{r}_2 \\
  B^{g}_2 \\
  B^{b}_2 \\
\end{pmatrix}
$$

we can rephrase the right side of the equation as:

$$
...
=
\begin{pmatrix}
  1 F^{r} + 0 F^{g} + 0 F^{b} - \alpha B^{r}_1\\
  0 F^{r} + 1 F^{g} + 0 F^{b} - \alpha B^{g}_1\\
  0 F^{r} + 0 F^{g} + 1 F^{b} - \alpha B^{b}_1\\
  1 F^{r} + 0 F^{g} + 0 F^{b} - \alpha B^{r}_2\\
  0 F^{r} + 1 F^{g} + 0 F^{b} - \alpha B^{g}_2\\
  0 F^{r} + 0 F^{g} + 1 F^{b} - \alpha B^{b}_2\\
\end{pmatrix}
=
\begin{pmatrix}
  1 & 0 & 0 & - B^{r}_1 \\
  0 & 1 & 0 & - B^{g}_1 \\
  0 & 0 & 1 & - B^{b}_1 \\
  1 & 0 & 0 & - B^{r}_2 \\
  0 & 1 & 0 & - B^{g}_2 \\
  0 & 0 & 1 & - B^{b}_2 \\
\end{pmatrix}
*
\begin{pmatrix}
	F^{r} \\
	F^{g} \\
	F^{b} \\
	\alpha \\
\end{pmatrix}
$$

Substituting this into the previous equations gives us the final result of

$$
\begin{pmatrix}
  C^{r}_1  -B^{r}_1 \\
  C^{g}_1 -B^{g}_1 \\
  C^{b}_1 -B^{b}_1 \\
  C^{r}_2 -B^{r}_2 \\
  C^{g}_2-B^{g}_2 \\
  C^{b}_2-B^{b}_2 \\
\end{pmatrix}=\begin{pmatrix}
1 & 	0 & 	0 & -B^{r}_1 \\
0 & 	1 & 	0 & -B^{g}_1 \\
0 & 	0 & 	1 & -B^{b}_1 \\
1 & 	0 & 	0 & -B^{r}_2 \\
0 & 	1 & 	0 & -B^{g}_2 \\
0 & 	0 & 	1 & -B^{b}_2 \\
\end{pmatrix} 
\begin{pmatrix}
F_r \\
F_g \\
F_b \\
\alpha
\end{pmatrix}
$$
\end{center}

\end{document}
